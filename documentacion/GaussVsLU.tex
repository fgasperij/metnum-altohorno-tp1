\subsection{Análisis de comparación Gauss contra LU}
Como se puede apreciar en los resultados obtenidos el tiempo de cómputo de Gauss fue menor que el tiempo de cómputo de LU en los casos en que se calculaba un solo vector $b$, sin embargo esta diferencia es relativamente pequeña porque el único extra de este algoritmo es la forward substitution.
Por el otro lado, al variar la cantidad de instancias el tiempo de cómputo de la factorización LU fue muy inferior al tiempo de la eliminación gaussiana, y esta diferencia se hace sumamente notoria a medida que la cantidad de vectores $b$ a calcular aumenta. Se pueden omitir del análisis los outliers iniciales que por algún motivo tardan más que el resto de los tests, posiblemente debido a que la caché no se encuentra preparada en la primera ejecución y por que la cantidad de instancias no es lo suficientemente alta para dara un valor del todo preciso.
