\subsection{Análisis de comparación Gauss contra LU}
Como se puede apreciar en los gráficos \ref{gaussVsLU1}, \ref{gaussVsLU2} y \ref{gaussVsLU3}, al
variar la cantidad de instancias el tiempo de cómputo de la factorización LU fue muy inferior al
tiempo de la eliminación gaussiana, y esta diferencia se hace sumamente notoria a medida que la
cantidad de vectores $b$ (a partir de ahora simplemente ``instancias'') a calcular aumenta. En el gráfico 
\ref{gaussVsLU4} se puede observar como la diferencia es proporcional tanto a la cantidad de instancias como al 
tamaño de las matrices, es decir, al aumentar la cantidad de instancias y/o el tamaño de la matriz LU aumenta 
la diferencia con respecto a Gauss. Por otro lado en \ref{gaussVsLU5} se puede determinar que basta con resolver el 
sistema de ecuacionescon sólo dos vectores $b$ distintos para poder ya apreciar una ventaja efectiva sobre la eliminación gaussiana 
y que para el caso en que solo haya un sólo vector $b$, los tiempos de cómputo son prácticamente idénticos, es decir la 
\emph{forward substitution} extra no agrega \emph{overhead} al cálculo de la solución. También es posible notar que para 
los diferentes tamaños matriciales, la relación se mantiene para una misma cantidad de instancias, y que dicha relación 
muestra una incremento significativo a medida que se se incrementa la cantidad de instancias, por lo que se puede determinar 
que el factor más importante que influye sobre la diferencia de rendimiento entre eliminación gaussiana es la cantidad de 
instancias y no tanto el tamaño de la matriz.
