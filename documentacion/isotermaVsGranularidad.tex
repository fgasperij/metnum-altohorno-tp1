\subsection{Análisis de comparación isoterma contra granularidad}
En el gráfico \ref{isotermaVsGranularidad} podemos ver claramente como en un principio aumentar un poco la granularidad 
nos cambia mucho el valor de la isoterma obtenida. Sin embargo, luego, a medida que la granularidad va en aumento, la 
isoterma va modificándose cada vez menos convergiendo a algún valor. Interpretamos que al aumentar la granularidad el 
valor obtenido para la isoterma converge a una magnitud, que corresponde a la isoterma real. Ésto coincide con la idea 
de que cuanta más precisión tengamos vamos a reducir el error de nuestros resultados. Creemos que la gran anomalía 
observada entre los primeros puntos del gráfico se debe a el reducido número de puntos utilizados en la 
discretización, esto provocaría un gran error en los resultados obtenidos con respecto al verdadero valor de la isoterma.
No sabemos certeramente a qué se debe la osilación presentada, sin embargo, creemos que simplemente se debe a cómo
se fue incrementando la discretización lo cual podría hacer que la isoterma se ubique de un lado u otro de el punto
de la discretización. Éste es un buen punto sobre el cual realizar otro experimento y contrastar los resultados.


