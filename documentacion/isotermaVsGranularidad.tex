\subsection{Análisis de comparación isoterma contra granularidad}
En el gráfico \ref{isotermaVsGranularidad} podemos ver claramente como en un principio aumentar un poco la granularidad nos cambia mucho el valor de la isoterma obtenida. Sin embargo, luego, a medida que la granularidad va en aumento, la isoterma va modificándose cada vez menos. Interpretamos que al aumentar la granularidad el valor obtenido para la isoterma converge a un valor, que corresponde a la isoterma real, porque cada vez tenemos más precisión y el error se reduce en la misma proporción que el aumento de la granularidad.

