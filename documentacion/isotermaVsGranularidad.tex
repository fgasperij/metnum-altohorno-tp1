\subsection{Análisis de comparación isoterma contra granularidad}
En el gráfico \ref{isotermaVsGranularidad} podemos ver claramente como en un principio aumentar un poco la granularidad nos cambia mucho el valor de la isoterma obtenida. Sin embargo, luego, a medida que la granularidad va en aumento, la isoterma va modificándose cada vez menos convergiendo a algún valor. Interpretamos que al aumentar la granularidad el valor obtenido para la isoterma converge a una magnitud, que corresponde a la isoterma real. Ésto coincide con la idea de que cuanta más precisión tengamos vamos a reducir el error de nuestros resultados.

