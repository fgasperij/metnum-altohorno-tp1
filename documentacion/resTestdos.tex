\subsection{Test de comparación Gauss vs LU}

A continuación mostramos los resultados obtenidos para el test de comparación entre factorización LU y eliminación gaussiana, los tiempos de cómputo se muestran en segundos. Se muestran los resultados de matrices de 2500x2500, 900x900 y 100x100. 


\begin{figure}[H]{}
\centering
\includegraphics[scale=0.5]{graphs/gaussVsLU1.pdf}
\caption{Resultados obtenidos usando matrices de 50 ángulos y 50 radios.}
\label{gaussVsLU1}
\end{figure}

\begin{figure}[H]{}
\centering
\includegraphics[scale=0.5]{graphs/gaussVsLU2.pdf}
\caption{Resultados obtenidos usando matrices de 30 ángulos y 30 radios.}
\label{gaussVsLU2}
\end{figure}

\begin{figure}[H]{}
\centering
\includegraphics[scale=0.5]{graphs/gaussVsLU3.pdf}
\caption{Resultados obtenidos usando matrices de 10 ángulos y 10 radios.}
\label{gaussVsLU3}
\end{figure}

\begin{figure}[H]{}
\centering
\includegraphics[scale=0.5]{graphs/dif.pdf}
\caption{Diferencia entre el tiempo consumido por LU y Gauss de los gráficos anteriores. Uso de escala logarítmica.}
\label{gaussVsLU4}
\end{figure}


\begin{figure}[H]{}
\centering
\includegraphics[scale=0.5]{graphs/ventRA.pdf}
\caption{Relación entre el tiempo total consumido por Gauss y LU en función de la cantidad de vectores b.}
\label{gaussVsLU5}
\end{figure}

\begin{figure}[H]{}
\centering
\includegraphics[scale=0.5]{graphs/gaussVsLU4.pdf}
\caption{Ventaja de Gauss sobre LU en función de la cantidad de vectores b para matrices de 10, 30 y 50 (ángulos y radios).}
\label{gaussVsLU6}
\end{figure}
