\section{Introducción Teórica}
%Contendrá una breve explicación de la base teórica que fundamenta los métodos involu-
%crados en el trabajo, junto con los métodos mismos. No deben incluirse demostraciones
%de propiedades ni teoremas, ejemplos innecesarios, ni definiciones elementales (como
%por ejemplo la de matriz simétrica). En vez de definiciones básicas es conveniente citar
%ejemplos de bibliografía adecuada. Una cita vale más que mil palabras.
%
El presente informe se enfoca la resolución de sistemas matriciales del tipo $Ax=b$, donde $A$ es una matriz inversible con coeficientes reales. El método clásico por excelencia para este tipo de problemas es la eliminación gaussiana que basicamente consiste en aplicar operaciones elementales de fila sobre la matriz $A$ y el vector $b$ para poder simplificar el sistema de ecuaciones original, obténiendose un sistema triangulado que puede ser fácilmente resuelto sustituyendo de manera correcta las incógnitas del sistema. La eliminación gaussiana original puede (y en algunas ocasiones debe) realizar permutaciones de filas, pero como vamos a restringir nuestro estudio a matrices diagonal dominantes (para más información, ver demostración en el apéndice), se omitirá el pivoteo, y cada vez que se haga mención a dicho algoritmo se entenderá que es sin pivoteo.
El otro método estudiado en este informe es la factorización LU, que básicamente consiste en hallar una forma de $A$ que sea igual a $L$ $\cdot$ $U$, donde $U$ es triangular superior y $L$ triangular inferior, de esta manera $Ax=b$ se convierte en $LUx=b$, y si consideramos $y=Ux$, la solución al sistema puede hallarse resolviendo primero $Ly=b$, y luego $Ux=y$. Esta factorización permite evitar tener que triangular la matriz $A$, cada vez que $b$ es modificado. La desventaja de este método es que la matriz debe poder triangularse usando eliminación gaussiana sin pivoteo, pero afortunadamente en este informe trabajaremos con dichas matrices.
