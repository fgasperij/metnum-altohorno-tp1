\section{Introducción Teórica}
%Contendrá una breve explicación de la base teórica que fundamenta los métodos involu- crados en el
%trabajo, junto con los métodos mismos. No deben incluirse demostraciones de propiedades ni
%teoremas, ejemplos innecesarios, ni definiciones elementales (como por ejemplo la de matriz
%simétrica). En vez de definiciones básicas es conveniente citar ejemplos de bibliografía adecuada.
%Una cita vale más que mil palabras.
%
En el presente trabajo estudiaremos diferentes métodos para la búsqueda de una isoterma en un la pared de un
horno industrial. Dado que computacionalmente es imposible analizar el problema para
funciones continuas lo discretizaremos. Los parámetros necesarios para calcular la isoterma con
el radio interno del horno, la cantidad de ángulos de la discretización, el radio externo del horno, 
la temperatura interna, la temperatura externa y la isoterma buscada. Se adjunta un gráfico para verlo con más claridad:

\begin{figure}[ht]
  \begin {center}
    \includegraphics[width=0.6\columnwidth]{Horno.png}
    \caption{Sección circular del horno}
  \end{center}
\end{figure}

Para lograrlo, nos enfocamos en la resolución de sistemas matriciales del tipo $Ax=b$,
donde $A$ es una matriz inversible con coeficientes reales. El método clásico por excelencia para
este tipo de problemas es la eliminación gaussiana que básicamente consiste en aplicar operaciones
elementales de fila sobre la matriz $A$ y el vector $b$ para poder simplificar el sistema de
ecuaciones original, obteniéndose un sistema triangulado que puede ser fácilmente resuelto aplicando
un algoritmo de sustitución hacia atrás \cite[6.1]{burden}. La eliminación gaussiana original puede
(y en algunas ocasiones debe) realizar permutaciones de filas, pero como vamos a restringir nuestro
estudio a matrices diagonal dominantes (para más información, ver demostración en el apéndice), se
omitirá el pivoteo, y cada vez que se haga mención a dicho algoritmo se entenderá que es sin
pivoteo.

El otro método estudiado en este informe es la factorización LU, que básicamente consiste en hallar
una forma de $A$ que sea igual a $L$ $\cdot$ $U$, donde $U$ es triangular superior y $L$ triangular
inferior, de esta manera $Ax=b$ se convierte en $LUx=b$, y si consideramos $y=Ux$, la solución al
sistema puede hallarse resolviendo primero $Ly=b$, y luego $Ux=y$. Esta factorización permite evitar
tener que triangular la matriz $A$, cada vez que $b$ es modificado \cite[6.5]{burden}.
