\subsection{Análisis de comparación de granularidad y tiempo}

Como era de esperarse, el tiempo de ejecución es creciente con la granularidad. Lo que sí
descubrimos en este test es que el tiempo parece crecer más que linealmente dado que la pendiente
de la función que parecerían describir los datos es mayor que la de la función lineal que usamos
para comparar. Esto en un principio parecería intuitivo dado que la complejidad de nuestros
algoritmos depende de la matriz A en la ecuación Ax=b, y las dimensiones de ésta dependen de la
granularidad, pero de la granularidad al cuadrado. Por ende, mientras más crece la granularidad más
crece la dimensión de la matriz, pero cuadráticamente.
