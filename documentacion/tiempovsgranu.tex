\subsection{Análisis de comparación de granularidad y tiempo}
Como era de esperarse, el tiempo de ejecución es creciente con la granularidad y, además,
aparentaría ser cúbico con una función con multiplicadores extraños. Esto no debería ser de
demasiada sorpresa dado que estamos midiendo en segundos y el tiempo es siempre menor a 15 segundos.
Si tenemos en cuenta esto y además que la granularidad va desde $30^2$ hasta $100^2$, no parece tan
alocado tener que dividir por $10^{11}$ y multiplicar por 2.

Éste fue un test satisfactorio dado que ocurrió lo que esperábamos: poder acotar el crecimiento de
los tiempos por una función cúbica y sin resultados extraños inesperados. Sí hay algunas pequeñas
irregularidades en los puntos, pero, aunque incomprobable, bien podría ser responsabilidad del
scheduler del sistema operativo u otra de todas las variables que podrían haber afectado mínimamente
los resultados, aún sacando promedio en 5 instancias.

% % Lo que sí
% descubrimos en este test es que el tiempo parece crecer más que linealmente dado que la pendiente
% de la función que parecerían describir los datos es mayor que la de la función lineal que usamos
% para comparar. Esto en un principio parecería intuitivo dado que la complejidad de nuestros
% algoritmos depende de la matriz A en la ecuación Ax=b, y las dimensiones de ésta dependen de la
% granularidad, pero de la granularidad al cuadrado. Por ende, mientras más crece la granularidad más
% crece la dimensión de la matriz, pero cuadráticamente.
