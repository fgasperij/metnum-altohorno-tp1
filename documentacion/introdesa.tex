\subsection{Algoritmos}

El objetivo principal del presente trabajo es resolver sistemas matriciales de la forma $Ax = b$, para el caso en que A sea una matriz inversible y diagonal dominante. Para poder resolver un sistema de ecuaciones en forma matricial, lo esencial es triangular la matriz para transformar el sistema, en principio complejo, en uno más simple que pueda ser resuelto mediante algún algoritmo sencillo.
Los métodos elegidos y estudiados para la triangulación del sistema son el algoritmo de eliminación de Gauss-Jordan sin pivoteo y la factorización LU, mientras que para resolver el sistema triangulado se usaron \emph{backward} y \emph{forward} \emph{substitution}. A continuación se muestran los pseudocódigos de los algoritmos implementados y la resolución de los sistemas de ecuaciones.

\begin{algorithm}[H]
\caption{gauss(Matriz $A$, vector $b$)}
\label{pseudo:gauss}
%\renewcommand\thealgorithm{}
\begin{algorithmic}
\FOR{$i=1$ hasta $n-1$}
	\IF{ $A_{ii} != 0$ }
		\FOR{$j=i+1$ hasta $n$}		
			\STATE $m = A_{ji}/A_{ii}$
			\FOR{$k=i$ hasta $n$}
				\STATE $A_{jk} = A_{jk} - m \cdot A_{ik}$
			\ENDFOR
			\STATE $b_{j} = b_{j} - m \cdot b_{i}$
		\ENDFOR
	\ENDIF
\ENDFOR
\end{algorithmic}
\end{algorithm}

\begin{algorithm}[H]
\caption{LU(Matriz $A$)}
\label{pseudo:lu}
%\renewcommand\thealgorithm{}
\begin{algorithmic}
\FOR{$i=1$ hasta $n$}
	\FOR{$j=i+1$ hasta $n-1$}		
		\IF{ $A_{ji} != 0$ }
			\STATE $m = A_{ji}/A_{ii}$
			\FOR{$k=i$ hasta $n$}
				\STATE $A_{jk} = A_{jk} - m \cdot A_{ik}$
			\ENDFOR
			\STATE $A_{ji} = m$
		\ENDIF
	\ENDFOR
\ENDFOR
\end{algorithmic}
\end{algorithm}

\begin{algorithm}[H]
\caption{forwSubst(Matriz $A$, vector $b$, vector $res$, bool $lu$)}
\label{pseudo:forwSubst}
%\renewcommand\thealgorithm{}
\begin{algorithmic}
\IF{$lu$}
	\FOR{$i=1$ hasta $n$}
		\STATE $auxVector$ = $A_{ii}$
		\STATE $A_{ii}$ = $1$
	\ENDFOR
\ENDIF
\FOR{$i=1$ hasta $n$}
	\STATE $acum$ $=$ $0$
	\FOR{$j=1$ hasta $j$ $<$ $i$}
		\STATE $acum += res_{j} \cdot A_{ij}$
	\ENDFOR
	\STATE $res_{i} = (b_{i} - acum)/A_{ii}$
\ENDFOR

\IF{$lu$}
	\FOR{$i=1$ hasta $n$}
		\STATE $A_{ii}$ = $auxVector$
	\ENDFOR
\ENDIF

\end{algorithmic}
\end{algorithm}

\begin{algorithm}[H]
\caption{backSubst(Matriz $A$, vector $b$, vector $res$, bool $lu$)}
\label{pseudo:backSubst}
%\renewcommand\thealgorithm{}
\begin{algorithmic}
\IF{$lu$}
	\FOR{$i=1$ hasta $n$}
		\STATE $auxVector$ = $A_{ii}$
		\STATE $A_{ii}$ = $1$
	\ENDFOR
\ENDIF
\FOR{$i=n$ hasta $1$}
	\STATE $acum$ $=$ $0$
	\FOR{$j=n$ hasta $j$ $>$ $i$}
		\STATE $acum += res_{j} \cdot A_{ij}$
	\ENDFOR
	\STATE $res_{i} = (b_{i} - acum)/A_{ii}$
\ENDFOR

\IF{$lu$}
	\FOR{$i=1$ hasta $n$}
		\STATE $A_{ii}$ = $auxVector$
	\ENDFOR
\ENDIF

\end{algorithmic}
\end{algorithm}

\begin{algorithm}[H]
\caption{resolverConGauss(Matriz $A$, vectores $bes$, vectores $reses$)}
\label{pseudo:resGauss}
%\renewcommand\thealgorithm{}
\begin{algorithmic}
\FOR{$i=1$ hasta $\#(bes)$}
	\STATE gauss($A$, $bes_{i}$)
	\STATE backSubst($A$, $bes_{i}$, $reses_{i}$, $false$)
\ENDFOR

\end{algorithmic}
\end{algorithm}

\begin{algorithm}[H]
\caption{resolverConLU(Matriz $A$, vectores $bes$, vectores $reses$)}
\label{pseudo:resLU}
%\renewcommand\thealgorithm{}
\begin{algorithmic}
\STATE LU($A$)
\FOR{$i=1$ hasta $\#(bes)$}
	\STATE backSubst($A$, $bes_{i}$, $aux$, $true$)
	\STATE forwSubst($A$, $aux$, $reses_{i}$, $true$)
\ENDFOR


\end{algorithmic}
\end{algorithm}





Aclaraciones:
\begin{itemize}
\item Como se puede observar en el pseudocódigo hay ciertas optimizaciones que no afectan a la correctitud de los algoritmos.
\item El código implementado permite usar pivoteo parcial, pero no será utilizado ni detallado en el informe.
\item La igualdad por cero está definida por la cercanía al cero de dicho número. Se usa una constante pequeña para decidir la igualdad.
\item El pseudocódigo presnta abusos de notación y es una mezcla de varios lenguajes de programación y lenguaje natural.
\end{itemize}







