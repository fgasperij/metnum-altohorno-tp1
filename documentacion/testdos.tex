\subsection{Test de comparación Gauss vs LU}
La siguiente experimentación tiene la intención de determinar las presuntas ventajas en determinadas condiciones de utilizar factorización LU en lugar del algoritmo de eliminación de Gauss-Jordan para hallar la (o las) solución (es) a un sistema de ecuaciones lineales. Para realizar dicha experimentación se generaron instancias aleatorias con las mismas semillas, utilizando los archivos (o modificaciones de los mismos) genTest.py y test.sh, y se compararon los tiempos de cómputo en función de la cantidad de puntos del sistema y la cantidad de instancias a resolver. El tiempo de cómputo de cada instancia y para cada algoritmo fue medido con los métodos provistos por la cátedra (ubicados en time.h). El tiempo de cómputo total para cada tamaño fue promediado como se aprecia en main.cpp. De esta manera si bien las instancias son ``aleatorias'', se usan las mismas tanto para LU como para Gauss. La idea de esta experimentación es determinar si al cambiar las condiciones del entorno (o en términos más el teóricos el vector $b$ del sistema $Ax=b$) continuamente, la factorización LU ahorra cálculos frente al método de eliminación de Gauss. Los únicos cálculos que fueron los correspondientes a los métodos de los algoritmos en cuestión más los necesarios para la resolución del sistema (back y forward substitution). Se omitió el tiempo de los métodos que plantean al sistema por no ser considerados parte de los algoritmos de resolución de sistemas matriciales.
