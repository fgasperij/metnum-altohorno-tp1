\subsubsection{Test de comparación Gauss vs LU}
La siguiente experimentación tiene la intención de determinar las presuntas ventajas en determinadas condiciones de utilizar factorización LU en lugar del algoritmo de eliminación de Gauss-Jordan para hallar la (o las) solución (es) a un sistema de ecuaciones lineales. Para realizar dicha experimentación se generaron instancias aleatorias con las mismas semillas, utilizando los archivos (o modificaciones de los mismos) \emph{genTest.py} y \emph{test.sh}, y se compararon los tiempos de cómputo en función de la cantidad de puntos del sistema y principalmente, la cantidad de instancias a resolver. 

Se realizaron varias ejecuciones de la experimentación considerando como valor valor final el promedio de dichas ejecuciones. Cabe aclarar que si bien las instancias son ``aleatorias'', se usan las mismas tanto para LU como para Gauss porque se usa la misma semilla. La idea de esta experimentación es determinar si al cambiar las condiciones del entorno (o en términos más el teóricos el vector $b$ del sistema $Ax=b$) en forma ``continua'', la factorización LU ahorra cálculos frente al método de eliminación de Gauss. Los tamaños de las matrices fueron fijados de manera que cubran el mayor espacio posible de instancias sin tener que caer en ejecuciones ``eternas''. Por otro lado, en todas las matrices la cantidad de radios y ángulos es la misma, para que los tiempos de cómputo sean lo más equilibrados posibles ya que en este experimentación no interesa mostrar comoa afecta la discretización al tiempo de cómputo.
