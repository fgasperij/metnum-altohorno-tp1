\section{Desarrollo}
%Deben explicarse los métodos numéricos que utilizaron y su aplicación al problema
%concreto involucrado en el trabajo práctico. Se deben mencionar los pasos que si-
%guieron para implementar los algoritmos, las dificultades que fueron encontrando y la
%descripción de cómo las fueron resolviendo. Explicar también cómo fueron planteadas
%y realizadas las mediciones experimentales. Los ensayos fallidos, hipótesis y conjeturas
%equivocadas, experimentos y métodos malogrados deben figurar en esta sección, con
%una breve explicación de los motivos de estas fallas (en caso de ser conocidas).

Decidimos pensar al problema como si un sistema lineal de ecuaciones o, 
equivalentemente, buscar el vector x que cumpla $Ax=b$, siendo éstas las siguientes:
\begin{compactitem}
  \item Matriz A: es una matriz cuadrada con cantidad de filas y de columnas igual a $n \times m$ 
  	está dividida en 3 partes según las filas.  Sean i,j tal que $1 \leq i,j \leq (n \times m)$. \\
  	\begin{compactitem}
	  \item \textbf{Caso} $i \leq n$ \textbf{ó Caso} $(n \times m) - n < i$:
	    \[ A_{ij} = \left\{ \begin{array}{ll}
               1 & \mbox{si $i = j$};\\
	       0 & \mbox{si $i \neq j$}.\end{array} \right. \] 
          \item \textbf{Caso} $n < i \leq (n \times m) - n$:
	    \[ A_{ij} = \left\{ \begin{array}{ll}
               \frac{-2}{(\Delta r)^2} + \frac{1}{r \times \Delta r} - \frac{2}{r^2 \times (\Delta \theta)^2}& \mbox{si $i = j$};\\ \ \\
               \frac{1}{(\Delta r)^2} - \frac{1}{r \times (\Delta r)}                                        & \mbox{si $j = i-n$};\\ \ \\
               \frac{1}{(\Delta r)^2}                                                                        & \mbox{si $j = i+n$};\\ \ \\
               \frac{1}{r^2 \times (\Delta \theta)^2}                                                        & \mbox{si $j = i-1$};\\ \ \\
               \frac{1}{r^2 \times (\Delta \theta)^2}                                                        & \mbox{si $j = i+1$};\\ \ \\
	       0 & \mbox{en otro caso}.\end{array} \right. \] 
	\end{compactitem}
 \item Vector x: es un vector con $n \times m$ incógnitas que representarían las temperaturas de los
 puntos en nuestra pared. Para que sea más fácil el cálculo y que sea consistente con lo propuesto
 en la matriz A, están ordenados de forma \textit{alfabética} primero según el radio (r) y después según
 el ángulo ($\theta$). Es decir, $X_1$ representa a T(1,1), $X_n$ representa a T(1,n), $X_{n+1}$ a
 T(2,1), etc.
 \item Vector b: es un vector con $n \times m$ valores y representarían lo que sabemos sobre las
 temperaturas. En el caso de los primeros n y últimos n valores, son las temperaturas internas y
 externas respectivamente. En los puntos intermedios entre ellos, todos los valores son 0. De esta
 forma, en las partes que son \item{submatrices inducidas} de A en las que hay una matriz
 \textbf{Identidad}, los 2 casos en la primera definición de $A_ij$ arriba, se igualaría el
 respectivo $X_i$ con su temperatura fija. En los puntos de la \textit{submatriz inducida de A} en
 los que no hay una Matriz \textbf{Identidad}, están igualados a 0 para aplicar la ecuación con
 derivadas con los multiplicadores de las incógnitas debidamente indicados por cada fila.
\end{compactitem}

Menos formalmente, sean $M_{i,j}, M_{i,i-n}, M_{i,i+n}, M_{i,i-1}$ y $ M_{i,i+1}$ los
multiplicadores en las filas de A ``del medio'' respectivamente según los enunciamos.

$
\begin{pmatrix}
  $I$    & \cdots    & 0      & \cdots     &   0     &   \cdots  &     \  &   0       &   \cdots   \\
  \vdots & \ddots    & \      & \cdots     &   0     &   \cdots  &     \  &         0 &   \cdots   \\ \hline
  \vdots & \vdots    & \      & \vdots     &   \     &   \vdots  &     \  &         \ &   \vdots   \\ 
  \vdots & M_{i,j-n} & \cdots &  M_{i,j-1} & M_{i,j} & M_{i,j+1} & \cdots & M_{i,j+n} & \cdots     \\
  \vdots & \vdots    & \      & \vdots     &   \     &   \vdots  &     \  &         \ &   \vdots   \\ \hline
  \cdots & \cdots    & 0      & \cdots     &   0     &   \cdots  &     \  &  I        &   \cdots   \\
  \cdots & \cdots    & 0      & \cdots     &   \     &   \cdots  &     \  &  \vdots        &   \ddots   \\ 

\end{pmatrix}
$

