\section{Desarrollo}
%Deben explicarse los métodos numéricos que utilizaron y su aplicación al problema
%concreto involucrado en el trabajo práctico. Se deben mencionar los pasos que si-
%guieron para implementar los algoritmos, las dificultades que fueron encontrando y la
%descripción de cómo las fueron resolviendo. Explicar también cómo fueron planteadas
%y realizadas las mediciones experimentales. Los ensayos fallidos, hipótesis y conjeturas
%equivocadas, experimentos y métodos malogrados deben figurar en esta sección, con
%una breve explicación de los motivos de estas fallas (en caso de ser conocidas).

Decidimos pensar al problema como si un sistema lineal de ecuaciones o, 
equivalentemente, buscar el vector x que cumpla $Ax=b$, siendo éstas las siguientes:
\begin{itemize}
  \item Matriz A: es una matriz cuadrada con cantidad de filas y de columnas igual a $n \times (m+1)$ 
  	está dividida en 3 partes según las filas.  Sean i,j tal que $1 \leq i,j \leq (n \times
	(m+1))$. \\
  	\begin{compactitem}
	  \item \textbf{Caso} $i \leq n$ \textbf{ó Caso} $(n \times (m+1)) - n < i$:
	    \[ A_{ij} = \left\{ \begin{array}{ll}
               1 & \mbox{si $i = j$};\\
	       0 & \mbox{si $i \neq j$}.\end{array} \right. \] 
          \item \textbf{Caso} $n < i \leq (n \times (m+1)) - n$:
	    \[ A_{ij} = \left\{ \begin{array}{ll}
               \frac{-2}{(\Delta r)^2} + \frac{1}{r \times \Delta r} - \frac{2}{r^2 \times (\Delta \theta)^2}& \mbox{si $i = j$};\\ \ \\
               \frac{1}{(\Delta r)^2} - \frac{1}{r \times (\Delta r)}                                        & \mbox{si $j = i-n$};\\ \ \\
               \frac{1}{(\Delta r)^2}                                                                        & \mbox{si $j = i+n$};\\ \ \\
               \frac{1}{r^2 \times (\Delta \theta)^2}                                                        & \mbox{si $j = i-1$};\\ \ \\
               \frac{1}{r^2 \times (\Delta \theta)^2}                                                        & \mbox{si $j = i+1$};\\ \ \\
	       0 & \mbox{en otro caso}.\end{array} \right. \] 
	\end{compactitem}
 \item Vector x: es un vector con $n \times (m+1)$ incógnitas que representarían las temperaturas de los
 puntos en nuestra pared. Para que sea más fácil el cálculo y que sea consistente con lo propuesto
 en la matriz A, están ordenados de forma \textit{alfabética} primero según el radio (r) y después según
 el ángulo ($\theta$). Es decir, $X_1$ representa a T(1,1), $X_n$ representa a T(1,n), $X_{n+1}$ a
 T(2,1), etc.
 \item Vector b: es un vector con $n \times (m+1)$ valores y representarían lo que sabemos sobre las
 temperaturas. En el caso de los primeros n y últimos n valores, son las temperaturas internas y
 externas respectivamente. En los puntos intermedios entre ellos, todos los valores son 0. De esta
 forma, en las partes que son \item{submatrices inducidas} de A en las que hay una matriz
 \textbf{Identidad}, los 2 casos en la primera definición de $A_ij$ arriba, se igualaría el
 respectivo $X_i$ con su temperatura fija. En los puntos de la \textit{submatriz inducida de A} en
 los que no hay una Matriz \textbf{Identidad}, están igualados a 0 para aplicar la ecuación con
 derivadas con los multiplicadores de las incógnitas debidamente indicados por cada fila.
\end{itemize}

Menos formalmente, sean $M_{i,j}, M_{i,i-n}, M_{i,i+n}, M_{i,i-1}$ y $ M_{i,i+1}$ los
multiplicadores en las filas de A ``del medio'' respectivamente según los enunciamos.

\begin{center}
$
\begin{pmatrix}
  $I$    & \cdots    & 0      & \cdots     &   0     &   \cdots  &     \  &   0       &   \cdots   \\
  \vdots & \ddots    & \      & \cdots     &   0     &   \cdots  &     \  &         0 &   \cdots   \\ \hline
  \vdots & \vdots    & \      & \vdots     &   \     &   \vdots  &     \  &         \ &   \vdots   \\ 
  \vdots & M_{i,j-n} & \cdots &  M_{i,j-1} & M_{i,j} & M_{i,j+1} & \cdots & M_{i,j+n} & \cdots     \\
  \vdots & \vdots    & \      & \vdots     &   \     &   \vdots  &     \  &         \ &   \vdots   \\ \hline
  \cdots & \cdots    & 0      & \cdots     &   0     &   \cdots  &     \  &  $I$        &   \cdots   \\
  \cdots & \cdots    & 0      & \cdots     &   \     &   \cdots  &     \  &  \vdots        &   \ddots   \\ 

\end{pmatrix}
$
\end{center}

\bigskip

\subsection{Algoritmos}

El objetivo principal del presente trabajo es resolver sistemas matriciales de la forma $Ax = b$, para el caso en que A sea una matriz inversible y diagonal dominante. Para poder resolver un sistema de ecuaciones en forma matricial, lo esencial es triangular la matriz para transformar el sistema, en principio complejo, en uno más simple que pueda ser resuelto mediante algún algoritmo sencillo.
Los métodos elegidos y estudiados para la triangulación del sistema son el algoritmo de eliminación de Gauss-Jordan sin pivoteo y la factorización LU, mientras que para resolver el sistema triangulado se usaron \emph{backward} y \emph{forward} \emph{substitution}. A continuación se muestran los pseudocódigos de los algoritmos implementados y la resolución de los sistemas de ecuaciones.

\begin{algorithm}[H]
\caption{gauss(Matriz $A$, vector $b$)}
\label{pseudo:gauss}
%\renewcommand\thealgorithm{}
\begin{algorithmic}
\FOR{$i=1$ hasta $n-1$}
	\IF{ $A_{ii} != 0$ }
		\FOR{$j=i+1$ hasta $n$}		
			\STATE $m = A_{ji}/A_{ii}$
			\FOR{$k=i$ hasta $n$}
				\STATE $A_{jk} = A_{jk} - m \cdot A_{ik}$
			\ENDFOR
			\STATE $b_{j} = b_{j} - m \cdot b_{i}$
		\ENDFOR
	\ENDIF
\ENDFOR
\end{algorithmic}
\end{algorithm}

\begin{algorithm}[H]
\caption{LU(Matriz $A$)}
\label{pseudo:lu}
%\renewcommand\thealgorithm{}
\begin{algorithmic}
\FOR{$i=1$ hasta $n$}
	\FOR{$j=i+1$ hasta $n-1$}		
		\IF{ $A_{ji} != 0$ }
			\STATE $m = A_{ji}/A_{ii}$
			\FOR{$k=i$ hasta $n$}
				\STATE $A_{jk} = A_{jk} - m \cdot A_{ik}$
			\ENDFOR
			\STATE $A_{ji} = m$
		\ENDIF
	\ENDFOR
\ENDFOR
\end{algorithmic}
\end{algorithm}

\begin{algorithm}[H]
\caption{forwSubst(Matriz $A$, vector $b$, vector $res$, bool $lu$)}
\label{pseudo:forwSubst}
%\renewcommand\thealgorithm{}
\begin{algorithmic}
\IF{$lu$}
	\FOR{$i=1$ hasta $n$}
		\STATE $auxVector$ = $A_{ii}$
		\STATE $A_{ii}$ = $1$
	\ENDFOR
\ENDIF
\FOR{$i=1$ hasta $n$}
	\STATE $acum$ $=$ $0$
	\FOR{$j=1$ hasta $j$ $<$ $i$}
		\STATE $acum += res_{j} \cdot A_{ij}$
	\ENDFOR
	\STATE $res_{i} = (b_{i} - acum)/A_{ii}$
\ENDFOR

\IF{$lu$}
	\FOR{$i=1$ hasta $n$}
		\STATE $A_{ii}$ = $auxVector$
	\ENDFOR
\ENDIF

\end{algorithmic}
\end{algorithm}

\begin{algorithm}[H]
\caption{backSubst(Matriz $A$, vector $b$, vector $res$, bool $lu$)}
\label{pseudo:backSubst}
%\renewcommand\thealgorithm{}
\begin{algorithmic}
\IF{$lu$}
	\FOR{$i=1$ hasta $n$}
		\STATE $auxVector$ = $A_{ii}$
		\STATE $A_{ii}$ = $1$
	\ENDFOR
\ENDIF
\FOR{$i=n$ hasta $1$}
	\STATE $acum$ $=$ $0$
	\FOR{$j=n$ hasta $j$ $>$ $i$}
		\STATE $acum += res_{j} \cdot A_{ij}$
	\ENDFOR
	\STATE $res_{i} = (b_{i} - acum)/A_{ii}$
\ENDFOR

\IF{$lu$}
	\FOR{$i=1$ hasta $n$}
		\STATE $A_{ii}$ = $auxVector$
	\ENDFOR
\ENDIF

\end{algorithmic}
\end{algorithm}

\begin{algorithm}[H]
\caption{resolverConGauss(Matriz $A$, vectores $bes$, vectores $reses$)}
\label{pseudo:resGauss}
%\renewcommand\thealgorithm{}
\begin{algorithmic}
\FOR{$i=1$ hasta $\#(bes)$}
	\STATE gauss($A$, $bes_{i}$)
	\STATE backSubst($A$, $bes_{i}$, $reses_{i}$, $false$)
\ENDFOR

\end{algorithmic}
\end{algorithm}

\begin{algorithm}[H]
\caption{resolverConLU(Matriz $A$, vectores $bes$, vectores $reses$)}
\label{pseudo:resLU}
%\renewcommand\thealgorithm{}
\begin{algorithmic}
\STATE LU($A$)
\FOR{$i=1$ hasta $\#(bes)$}
	\STATE backSubst($A$, $bes_{i}$, $aux$, $true$)
	\STATE forwSubst($A$, $aux$, $reses_{i}$, $true$)
\ENDFOR


\end{algorithmic}
\end{algorithm}





Aclaraciones:
\begin{itemize}
\item Como se puede observar en el pseudocódigo hay ciertas optimizaciones que no afectan a la correctitud de los algoritmos.
\item El código implementado permite usar pivoteo parcial, pero no será utilizado ni detallado en el informe.
\item La igualdad por cero está definida por tolerancia.
\item El pseudocódigo presenta abusos de notación y es una mezcla de varios lenguajes de programación y lenguaje natural.
\item Algunos algoritmos pueden estar implementados en un mismo método.
\item La clase matriz es básicamente un vector de vectores, más un vector que va guardando rastro de las permutaciones. Como la eliminación gaussiana con pivoteo parcial no va a ser estudiada en el informe, ya que se privilegió la performance de los algoritmos, esta último vector puede ser omitido.
\end{itemize}









DESCRIPCION TESTS
Una vez detallado los algoritmos se pasó a la etapa de experimentación, realizandose los siguientes tests:
*********************************** TEST UNO	**********************


%\subsubsection{Test de isoterma en función de la granularidad de la discretización}
La siguiente experimentación tiene la intención de analizar el comportamiento de la isoterma en función de la granularidad de la discretización.
Tomaremos $r_i$, $r_e$, $T_i(\theta_j)$, $T_e(\theta_j)$ e $iso$ constantes y apropiados para que se pueda apreciar con mayor claridad el comportamiento, la elección de los mismos fue tomada luego de varias pruebas.
La cantidad de ángulos y radios tomada es la misma en todas las instancias, $m = n$. El objetivo es poder aislar el factor granularidad y ver de qué forma este afecta a la isoterma obtenida.

\subsection{Test de comparación GAUSS vs LU}


