\section{Desarrollo}
%Deben explicarse los métodos numéricos que utilizaron y su aplicación al problema
%concreto involucrado en el trabajo práctico. Se deben mencionar los pasos que si-
%guieron para implementar los algoritmos, las dificultades que fueron encontrando y la
%descripción de cómo las fueron resolviendo. Explicar también cómo fueron planteadas
%y realizadas las mediciones experimentales. Los ensayos fallidos, hipótesis y conjeturas
%equivocadas, experimentos y métodos malogrados deben figurar en esta sección, con
%una breve explicación de los motivos de estas fallas (en caso de ser conocidas).

Decidimos pensar al problema como un sistema lineal de ecuaciones o, 
equivalentemente, buscar el vector x que cumpla $Ax=b$, siendo éstas las siguientes:
\begin{itemize}
  \item Matriz A: es una matriz cuadrada con cantidad de filas y de columnas igual a $n \times (m+1)$ 
  	está dividida en 3 partes según las filas.  Sean i,j tal que $1 \leq i,j \leq (n \times
	(m+1))$. \\
  	\begin{compactitem}
	  \item \textbf{Caso} $i \leq n$ \textbf{ó Caso} $(n \times (m+1)) - n < i$:
	    \[ A_{ij} = \left\{ \begin{array}{ll}
               1 & \mbox{si $i = j$};\\
	       0 & \mbox{si $i \neq j$}.\end{array} \right. \] 
          \item \textbf{Caso} $n < i \leq (n \times (m+1)) - n$:
	    \[ A_{ij} = \left\{ \begin{array}{ll}
               \frac{-2}{(\Delta r)^2} + \frac{1}{r \times \Delta r} - \frac{2}{r^2 \times (\Delta \theta)^2}& \mbox{si $i = j$};\\ \ \\
               \frac{1}{(\Delta r)^2} - \frac{1}{r \times (\Delta r)}                                        & \mbox{si $j = i-n$};\\ \ \\
               \frac{1}{(\Delta r)^2}                                                                        & \mbox{si $j = i+n$};\\ \ \\
               \frac{1}{r^2 \times (\Delta \theta)^2}                                                        & \mbox{si $j = i-1$};\\ \ \\
               \frac{1}{r^2 \times (\Delta \theta)^2}                                                        & \mbox{si $j = i+1$};\\ \ \\
	       0 & \mbox{en otro caso}.\end{array} \right. \] 
	\end{compactitem}
 \item Vector x: es un vector con $n \times (m+1)$ incógnitas que representarían las temperaturas de los
 puntos en nuestra pared. Para que sea más fácil el cálculo y que sea consistente con lo propuesto
 en la matriz A, están ordenados de forma \textit{alfabética} primero según el radio (r) y después según
 el ángulo ($\theta$). Es decir, $X_1$ representa a T(1,1), $X_p$ representa a T(p/n,p\%n), $X_{n+1}$ a
 T((p+1)/n,(p+1)\%n), etc.
 \item Vector b: es un vector con $n \times (m+1)$ valores que representan lo que sabemos sobre las
 temperaturas, es decir, las temperaturas internas y externas y el resultado de las ecuaciones de calor.
 Los primeros y últimos n valores son las temperaturas internas y externas respectivamente. Los puntos 
 intermedios entre ellos son todos 0s, el resultado de la ecuación de calor para ese punto. De esta forma, 
 queda subdividido en las siguientes partes: \item{submatrices inducidas} de A en las que hay una matriz
 \textbf{Identidad}, los 2 casos en la primera definición de $A_ij$ arriba, se igualaría el
 respectivo $X_i$ con su temperatura fija. En los puntos de la \textit{submatriz inducida de A} en
 los que no hay una Matriz \textbf{Identidad}, están igualados a 0 para aplicar la ecuación con
 derivadas con los multiplicadores de las incógnitas debidamente indicados por cada fila.
\end{itemize}

Menos formalmente, sean $M_{i,j}, M_{i,i-n}, M_{i,i+n}, M_{i,i-1}$ y $ M_{i,i+1}$ los
multiplicadores en las filas de A ``del medio'' respectivamente según los enunciamos.

\begin{center}
$
\begin{pmatrix}
  $I$    & \cdots    & 0      & \cdots     &   0     &   \cdots  &     \  &   0       &   \cdots   \\
  \vdots & \ddots    & \      & \cdots     &   0     &   \cdots  &     \  &         0 &   \cdots   \\ \hline
  \vdots & \vdots    & \      & \vdots     &   \     &   \vdots  &     \  &         \ &   \vdots   \\ 
  \vdots & M_{i,j-n} & \cdots &  M_{i,j-1} & M_{i,j} & M_{i,j+1} & \cdots & M_{i,j+n} & \cdots     \\
  \vdots & \vdots    & \      & \vdots     &   \     &   \vdots  &     \  &         \ &   \vdots   \\ \hline
  \cdots & \cdots    & 0      & \cdots     &   0     &   \cdots  &     \  &  $I$        &   \cdots   \\
  \cdots & \cdots    & 0      & \cdots     &   \     &   \cdots  &     \  &  \vdots        &   \ddots   \\ 

\end{pmatrix}
$
\end{center}

\bigskip

\subsection{Algoritmos}

El objetivo principal del presente trabajo es resolver sistemas matriciales de la forma $Ax = b$, para el caso en que A sea una matriz inversible y diagonal dominante. Para poder resolver un sistema de ecuaciones en forma matricial, lo esencial es triangular la matriz para transformar el sistema, en principio complejo, en uno más simple que pueda ser resuelto mediante algún algoritmo sencillo.
Los métodos elegidos y estudiados para la triangulación del sistema son el algoritmo de eliminación de Gauss-Jordan sin pivoteo y la factorización LU, mientras que para resolver el sistema triangulado se usaron \emph{backward} y \emph{forward} \emph{substitution}. A continuación se muestran los pseudocódigos de los algoritmos implementados y la resolución de los sistemas de ecuaciones.

\begin{algorithm}[H]
\caption{gauss(Matriz $A$, vector $b$)}
\label{pseudo:gauss}
%\renewcommand\thealgorithm{}
\begin{algorithmic}
\FOR{$i=1$ hasta $n-1$}
	\IF{ $A_{ii} != 0$ }
		\FOR{$j=i+1$ hasta $n$}		
			\STATE $m = A_{ji}/A_{ii}$
			\FOR{$k=i$ hasta $n$}
				\STATE $A_{jk} = A_{jk} - m \cdot A_{ik}$
			\ENDFOR
			\STATE $b_{j} = b_{j} - m \cdot b_{i}$
		\ENDFOR
	\ENDIF
\ENDFOR
\end{algorithmic}
\end{algorithm}

\begin{algorithm}[H]
\caption{LU(Matriz $A$)}
\label{pseudo:lu}
%\renewcommand\thealgorithm{}
\begin{algorithmic}
\FOR{$i=1$ hasta $n$}
	\FOR{$j=i+1$ hasta $n-1$}		
		\IF{ $A_{ji} != 0$ }
			\STATE $m = A_{ji}/A_{ii}$
			\FOR{$k=i$ hasta $n$}
				\STATE $A_{jk} = A_{jk} - m \cdot A_{ik}$
			\ENDFOR
			\STATE $A_{ji} = m$
		\ENDIF
	\ENDFOR
\ENDFOR
\end{algorithmic}
\end{algorithm}

\begin{algorithm}[H]
\caption{forwSubst(Matriz $A$, vector $b$, vector $res$, bool $lu$)}
\label{pseudo:forwSubst}
%\renewcommand\thealgorithm{}
\begin{algorithmic}
\IF{$lu$}
	\FOR{$i=1$ hasta $n$}
		\STATE $auxVector$ = $A_{ii}$
		\STATE $A_{ii}$ = $1$
	\ENDFOR
\ENDIF
\FOR{$i=1$ hasta $n$}
	\STATE $acum$ $=$ $0$
	\FOR{$j=1$ hasta $j$ $<$ $i$}
		\STATE $acum += res_{j} \cdot A_{ij}$
	\ENDFOR
	\STATE $res_{i} = (b_{i} - acum)/A_{ii}$
\ENDFOR

\IF{$lu$}
	\FOR{$i=1$ hasta $n$}
		\STATE $A_{ii}$ = $auxVector$
	\ENDFOR
\ENDIF

\end{algorithmic}
\end{algorithm}

\begin{algorithm}[H]
\caption{backSubst(Matriz $A$, vector $b$, vector $res$, bool $lu$)}
\label{pseudo:backSubst}
%\renewcommand\thealgorithm{}
\begin{algorithmic}
\IF{$lu$}
	\FOR{$i=1$ hasta $n$}
		\STATE $auxVector$ = $A_{ii}$
		\STATE $A_{ii}$ = $1$
	\ENDFOR
\ENDIF
\FOR{$i=n$ hasta $1$}
	\STATE $acum$ $=$ $0$
	\FOR{$j=n$ hasta $j$ $>$ $i$}
		\STATE $acum += res_{j} \cdot A_{ij}$
	\ENDFOR
	\STATE $res_{i} = (b_{i} - acum)/A_{ii}$
\ENDFOR

\IF{$lu$}
	\FOR{$i=1$ hasta $n$}
		\STATE $A_{ii}$ = $auxVector$
	\ENDFOR
\ENDIF

\end{algorithmic}
\end{algorithm}

\begin{algorithm}[H]
\caption{resolverConGauss(Matriz $A$, vectores $bes$, vectores $reses$)}
\label{pseudo:resGauss}
%\renewcommand\thealgorithm{}
\begin{algorithmic}
\FOR{$i=1$ hasta $\#(bes)$}
	\STATE gauss($A$, $bes_{i}$)
	\STATE backSubst($A$, $bes_{i}$, $reses_{i}$, $false$)
\ENDFOR

\end{algorithmic}
\end{algorithm}

\begin{algorithm}[H]
\caption{resolverConLU(Matriz $A$, vectores $bes$, vectores $reses$)}
\label{pseudo:resLU}
%\renewcommand\thealgorithm{}
\begin{algorithmic}
\STATE LU($A$)
\FOR{$i=1$ hasta $\#(bes)$}
	\STATE backSubst($A$, $bes_{i}$, $aux$, $true$)
	\STATE forwSubst($A$, $aux$, $reses_{i}$, $true$)
\ENDFOR


\end{algorithmic}
\end{algorithm}





Aclaraciones:
\begin{itemize}
\item Como se puede observar en el pseudocódigo hay ciertas optimizaciones que no afectan a la correctitud de los algoritmos.
\item El código implementado permite usar pivoteo parcial, pero no será utilizado ni detallado en el informe.
\item La igualdad por cero está definida por tolerancia.
\item El pseudocódigo presnta abusos de notación y es una mezcla de varios lenguajes de programación y lenguaje natural.
\item Algunos algoritmos pueden estar implementados en un mismo método.
\end{itemize}









%DESCRIPCION TESTS
%Una vez detallado los algoritmos se pasó a la etapa de experimentación, realizandose los siguientes tests:
%*********************************** TEST UNO	**********************

\subsection{Tests}
En esta sección haremos una introducción a los diferentes tests que decidimos hacer en nuestro
trabajo.
\subsubsection{Test de isoterma en función de la granularidad de la discretización}
% La siguiente experimentación tiene la intención de analizar el comportamiento de la isoterma en función de la
% granularidad de la discretización.  Tomaremos $r_i$, $r_e$, $T_i(\theta_j)$, $T_e(\theta_j)$ e $iso$
% constantes y apropiados para que se pueda apreciar con mayor claridad el comportamiento, la elección
% de los mismos fue tomada luego de varias pruebas.  La cantidad de ángulos y radios tomada es la
% misma en todas las instancias, $m = n$. El objetivo es poder aislar el factor granularidad y ver de
% qué forma este afecta a la isoterma obtenida.

En nuestra primera experimentación, analizaremos cómo iría cambiando la posición de la isoterma en
función de la granularidad de la discretización. Para ello, fijaremos a $r_i$, $r_e$,
$T_i(\theta_j)$, $T_e(\theta_j)$ e $iso$ para poder concentrarnos sólo en la isoterma\footnote{Los
  valores están especificados abajo y fueron elegidos luego de varios tests}
%j
  DEEEEEEEEEEEEEEEEEEECIRRRRRRRR CUÁAAAAALES SOOOOOOOON ESTOOOOS VALOOOOOORES Y POR QUÉEEEEEEEEEE.
  %%
Otra decisión fue tomar a la cantidad de ángulos y la cantidad de radios iguales para cada
instancia, es decir, $n=m$, para facilitar el test.  Esto significaría que nuestra matriz será
cuadrada con dimensiones $\mathbb{R}^{n*(n+1) \times n*(n+1)}$. Se podría testear en un futuro las
implicancias de variar ambos al mismo tiempo. También para tener en cuenta es el hecho que la matriz
es una \textit{matriz banda} dado que en cada fila $i$ se tienen sólo 5 datos no nulos, y todos
ellos entre las columnas $j-n$ y $j+n$. Esto último significaría que los datos a guardados por cada
fila son mucho mayores a los realmente usados, por lo que en alguna futura expansión de este trabajo
se podría adaptar el algoritmo a este hecho y disminuir su complejidad espacial rotundamente.

Finalmente, como las temperaturas externas e internas están fijas, no es necesario calcular la
isoterma por cada ángulo dado que estarán siempre en el mismo radio para cada uno de ellos. Esto se
debe a que entre los ángulos no hay diferencia, y según las propiedades del calor enunciadas por la
cátedra éste se comportaría igual en cada uno de los ángulos.


ACÁ PONER LOS VALORES ELEGIDOOOOOOOOOOOOOOOOOOOOOOOOOOOOOOOOOOOOOOOOOOOOOOOOOOS


Sobre este test esperamos que a medida aumente la granularidad, la isoterma vaya tendiendo a un
punto dado que aumentaríamos la precisión con la cual la buscamos. Parece intuitivo que mientras más
extensa sea la \textit{grilla} por la cual dividimos al horno, más nos estaríamos acercando al caso
real continuo.

\subsubsection{Test de comparación Gauss vs LU}
La siguiente experimentación tiene la intención de determinar las presuntas ventajas en determinadas condiciones de utilizar factorización LU en lugar del algoritmo de eliminación de Gauss-Jordan para hallar la (o las) solución (es) a un sistema de ecuaciones lineales. Para realizar dicha experimentación se generaron instancias aleatorias con las mismas semillas, utilizando los archivos (o modificaciones de los mismos) \emph{genTest.py} y \emph{test.sh}, y se compararon los tiempos de cómputo en función de la cantidad de puntos del sistema y principalmente, la cantidad de instancias a resolver. 

Se realizaron varias ejecuciones de la experimentación considerando como valor valor final el promedio de dichas ejecuciones. Cabe aclarar que si bien las instancias son ``aleatorias'', se usan las mismas tanto para LU como para Gauss porque se usa la misma semilla. La idea de esta experimentación es determinar si al cambiar las condiciones del entorno (o en términos más el teóricos el vector $b$ del sistema $Ax=b$) en forma ``continua'', la factorización LU ahorra cálculos frente al método de eliminación de Gauss. Los tamaños de las matrices fueron fijados de manera que cubran el mayor espacio posible de instancias sin tener que caer en ejecuciones ``eternas''. Por otro lado, en todas las matrices la cantidad de radios y ángulos es la misma, para que los tiempos de cómputo sean lo más equilibrados posibles. Más adelante se mostrará como afecta la discretización al tiempo de cómputo.

\subsection{Test de isoterma en función de las condiciones de borde}
La siguiente experimentación tiene por objetivo principal analizar cómo se comporta el modelo implementado. Es decir, ver cómo refleja los cambios esperados en un situación dada del problema modelado.
En un altohorno real $r_i$ y $r_e$ no sufren cambios, lo que varía constantente son las $T_i(\theta_j)$ y las $T_e(\theta_j)$. Para simular esto y poder ver cómo varía la isoterma elegimos una buena discretización, la mayor que podamos costear en función de nuestra potencia de procesamiento, mantenemos $r_i$ y $r_e$ constantes y vamos variando las $T_i(\theta_j)$ y las $T_e(\theta_j)$ de diferentes formas para ver cómo se comporta la isoterma.
Los casos que vamos a considerar son los siguientes:
\begin{enumerate}
 \item aumenta $T_i(\theta_j)$ y $T_e(\theta_j)$ para todo $j$ de la discretización.
 \item aumenta $T_i(\theta_j)$ y se mantienen constante $T_e(\theta_j)$ para todo $j$ de la discretización.
 \item aumenta $T_i(\theta_j)$ y desciende $T_e(\theta_j)$ para todo $j$ de la discretización.
\end{enumerate}
Elegimos estos tres casos porque creemos que cubren la mayoría de los comportamientos. A nivel teórico no hacemos distinción entre las $T_i(\theta_j)$ y las $T_e(\theta_j)$, por lo tanto tenemos dos componentes que pueden cambiar en un sentido u otro o mantenerse constantes. El caso 1 tomas que las dos componentes cambian y en el mismo sentido. El caso 2 que una componente cambia y la otra permanece constante. Finalmente, el caso 3 considera la situación en la que las dos componentes cambian pero en sentidos inversos.

\subsubsection{Test de tiempo en función de la granularidad de la discretización}
La siguiente experimentación tiene por objetivo analizar cómo y cuánto influye la granularidad de la
discretización en el tiempo de ejecución de nuestros algoritmos. Sólo utilizamos el método de
Gauss-Jordan ya que no nos interesa comparar Gauss-Jordan con LU, hay otro test específico para eso.

Para aislar la componente granularidad decidimos tomar $r_i$, $r_e$, $T_i(\theta_j)$,
$T_e(\theta_j)$ e $iso$ constantes. Lo único que varía entre las instancias del test es la cantidad
de ángulos y radios en la que discretizamos el horno. Tomamos cantidades de ángulos y radios siempre
iguales porque sólo nos interesa la granularidad como cantidad de puntos. Es decir, que la cantidad
de puntos en nuestra discretización va a ser:
\begin{equation}
 puntos = m^2 = n^2
\end{equation}

Nuestro objetivo es comprobar que la complejidad temporal empírica es la esperada por un algoritmo
de eliminación Gaussiana de O($n^3$). Además, dado que es un algoritmo en el cual los ciclos son
siempre exhaustivos\footnote{Si miramos el pseudocódigo del algoritmo en la sección desarrollo,
veremos que siempre entra a los \textit{fors} y no sale hasta iterar todos los \textit{i}.}, no esperamos saltos
demasiado agitantados entre un tiempo y el siguiente.

T
