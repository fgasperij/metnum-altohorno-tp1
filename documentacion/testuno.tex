\subsubsection{Test de isoterma en función de la granularidad de la discretización}
% La siguiente experimentación tiene la intención de analizar el comportamiento de la isoterma en función de la
% granularidad de la discretización.  Tomaremos $r_i$, $r_e$, $T_i(\theta_j)$, $T_e(\theta_j)$ e $iso$
% constantes y apropiados para que se pueda apreciar con mayor claridad el comportamiento, la elección
% de los mismos fue tomada luego de varias pruebas.  La cantidad de ángulos y radios tomada es la
% misma en todas las instancias, $m = n$. El objetivo es poder aislar el factor granularidad y ver de
% qué forma este afecta a la isoterma obtenida.

En nuestra primera experimentación, analizaremos cómo iría cambiando la posición de la isoterma en
función de la granularidad de la discretización. Para ello, fijaremos a $r_i$, $r_e$,
$T_i(\theta_j)$, $T_e(\theta_j)$ e $iso$ para poder concentrarnos sólo en la isoterma\footnote{Los
  valores están especificados abajo y fueron elegidos luego de varios tests}
%j
  DEEEEEEEEEEEEEEEEEEECIRRRRRRRR CUÁAAAAALES SOOOOOOOON ESTOOOOS VALOOOOOORES Y POR QUÉEEEEEEEEEE.
  %%
Otra decisión fue tomar a la cantidad de ángulos y la cantidad de radios iguales para cada
instancia, es decir, $n=m$, para facilitar el test.  Esto significaría que nuestra matriz será
cuadrada con dimensiones $\mathbb{R}^{n*(n+1) \times n*(n+1)}$. Se podría testear en un futuro las
implicancias de variar ambos al mismo tiempo. También para tener en cuenta es el hecho que la matriz
es una \textit{matriz banda} dado que en cada fila $i$ se tienen sólo 5 datos no nulos, y todos
ellos entre las columnas $j-n$ y $j+n$. Esto último significaría que los datos a guardados por cada
fila son mucho mayores a los realmente usados, por lo que en alguna futura expansión de este trabajo
se podría adaptar el algoritmo a este hecho y disminuir su complejidad espacial rotundamente.

Finalmente, como las temperaturas externas e internas están fijas, no es necesario calcular la
isoterma por cada ángulo dado que estarán siempre en el mismo radio para cada uno de ellos. Esto se
debe a que entre los ángulos no hay diferencia, y según las propiedades del calor enunciadas por la
cátedra éste se comportaría igual en cada uno de los ángulos.


ACÁ PONER LOS VALORES ELEGIDOOOOOOOOOOOOOOOOOOOOOOOOOOOOOOOOOOOOOOOOOOOOOOOOOOS
