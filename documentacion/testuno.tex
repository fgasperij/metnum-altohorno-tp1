\subsubsection{Test de isoterma en función de la granularidad de la discretización}
La siguiente experimentación tiene la intención de analizar el comportamiento de la isoterma en función de la granularidad de la discretización.
Tomaremos $r_i$, $r_e$, $T_i(\theta_j)$, $T_e(\theta_j)$ e $iso$ constantes y apropiados para que se pueda apreciar con mayor claridad el comportamiento, la elección de los mismos fue tomada luego de varias pruebas.
La cantidad de ángulos y radios tomada es la misma en todas las instancias, $m = n$. El objetivo es poder aislar el factor granularidad y ver de qué forma este afecta a la isoterma obtenida.
