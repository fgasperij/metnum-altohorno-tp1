\subsubsection{Test de isoterma en función de la granularidad de la discretización}
% La siguiente experimentación tiene la intención de analizar el comportamiento de la isoterma en función de la
% granularidad de la discretización.  Tomaremos $r_i$, $r_e$, $T_i(\theta_j)$, $T_e(\theta_j)$ e $iso$
% constantes y apropiados para que se pueda apreciar con mayor claridad el comportamiento, la elección
% de los mismos fue tomada luego de varias pruebas.  La cantidad de ángulos y radios tomada es la
% misma en todas las instancias, $m = n$. El objetivo es poder aislar el factor granularidad y ver de
% qué forma este afecta a la isoterma obtenida.

En nuestra primera experimentación, analizaremos el comportamiento de la posición de la isoterma en
función de la granularidad de la discretización utilizada. Para ello, fijaremos a los valores de$r_i$, $r_e$,
$T_i(\theta_j)$, $T_e(\theta_j)$ e $iso$ para poder aislar el comportamiento de la isoterma\footnote{Los
  valores están especificados abajo y fueron elegidos luego de varios tests}
Además, con la misma motivación, tomamos una cantidad de ángulos y radios iguales para cada
instancia, es decir, $n=m$, para facilitar el test.  Esto representa que la matriz resultante será
cuadrada y de dimensiones $\mathbb{R}^{n*(n+1) \times n*(n+1)}$. No vamos a considerar el caso
en el que la cantidad de ángulos y radios varíe de forma distinta. Es destacable el hecho de que la matriz
es una \textit{matriz banda} dado que en cada fila $i$ se tienen sólo 5 datos no nulos, y todos
ellos entre las columnas $j-n$ y $j+n$. Esto último significa que los datos guardados por cada
fila son mucho mayores a los realmente usados, por lo que una mejora posible sería adaptar el algoritmo a 
este hecho y disminuir su complejidad espacial rotundamente.
Finalmente, queremos aclarar que al estar las temperaturas externas e internas fijas, no es necesario calcular la
isoterma por cada ángulo dado que lo anterior implica que no variará. Esto se
debe a que entre los ángulos no hay diferencia, y según las propiedades del calor enunciadas por la
cátedra éste se comportaría igual en cada uno de los ángulos.
Esperamos que el aumento de la granularidad en la discretización produzca que el valor de la isoterma converja
ya que aumentar la granularidad es equivalente a aumentar la precisión de la búsqueda. Parece intuitivo que mientras más
extensa sea la \textit{grilla} por la cual dividimos al horno, más nos estaríamos acercando al caso
real continuo.
