
\begin{figure}[ptb]
\includegraphics[scale=0.30]{logo.jpg}\hspace{6cm}
\includegraphics[scale=0.90]{logo_dc.jpg}
\end{figure}

\newcommand{\autores}{Bartalotta, Danós, Gasperi Jabalera, Sarriés}

%Datos de la caratula
\materia{M\'etodos Num\'ericos}
\titulo{Trabajo Pr\'actico 1}
\subtitulo{Con 15 thetas discretizo alto horno}
\hspace{6cm}
\integrante{Bartalotta, Franco}{347/11}{}
\integrante{Danós, Alejandro}{381/10}{adp007@gmail.com}
\integrante{Gasperi Jabalera, Fernando}{56/09}{yolibertino@gmail.com}
\integrante{Sarriés, Ana}{144/02}{abarloventos@gmail.com}
\resumen{En este trabajo estudiaremos algoritmos para resolver sistemas de ecuaciones para problemas
reales. Se expondrán técnicas para obtener alguna isoterma buscada en un muro de concreto de un
horno usando eliminación Gaussiana y Factorización LU. Se estudiará el rendimiento de ambos y las
ventajas de cada uno. Se analizarán los efectos en el cambio de granularización del horno. Al final
del trabajo, se llegarán a conclusiones sobre lo descubierto.}
\palabrasClave{Eliminación Gaussiana. Factorización LU. Iosterma. Horno. Matrices. Sistemas de
  ecuaciones. Complejidad temporal.}
\hypersetup{%
 % Para que el PDF se abra a página completa.
 pdfstartview= {FitH \hypercalcbp{\paperheight-\topmargin-1in-\headheight}},
 pdfauthor={\autores},
 pdfsubject={TP1}
}

\parskip=5pt % 10pt es el tamaño de fuente

% Pongo en 0 la distancia extra entre ítemes.
\let\olditemize\itemize
\def\itemize{\olditemize\itemsep=0pt}

% Acomodo fancyhdr <- Creo que es el encabezado de pagina
\pagestyle{fancy}
\thispagestyle{fancy}
\addtolength{\headheight}{1pt}
\lhead{M\'etodos Num\'ericos - 1$^{er}$ Cuatr. 2014}
\rhead{RTP1 - \autores}
\cfoot{\thepage}
\renewcommand{\footrulewidth}{0.4pt}




%Pagina de titulo e indice
\thispagestyle{empty}

\maketitle
\tableofcontents

\newpage

