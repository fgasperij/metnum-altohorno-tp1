\subsubsection{Demostración de la proposición}
Proposición:

Sea $A \in K^{n \times n}$ con $K$ $=$ $\mathbb{R}$ o $\mathbb{C}$ una matriz diagonal dominante e inversible entonces es posible aplicar eliminación gaussiana sin pivoteo.

Notación útil: 

$A(i|j)$ := Es la submatriz de $A$ que se obtiene de elimar las filas de $1$ a $i$ y las columnas de $1$ a $j$.

$A^{(k)}$ := Es la matriz obtenida luego de relizar $k$ pasos de eliminación gaussiana sobre las filas de $A$.

Demostración:

Como A es diagonal dominante, se tiene que $|A_{ii}|$ $\geq$ $\sum_{j=1, i \neq j}^{n}$ $|A_{ij}|$ para todo $j$ $\neq$ $i$, por lo tanto debe suceder que $A_{11}$ $\neq$ $0$, si así no fuera, se tiene que $0$ $=$ $|A_{11}|$ $\geq$ $|A_{1j}|$  $\geq$ $0$ para todo $j$ $\neq$ $2$ a $n$, por lo que $|A_{1j}|$ $=$ $0$, y esto sucede si y solo si $A_{ij}$ $=$ $0$, es decir, la columna $1$ es nula y por lo tanto A no es inversible, pero A es inversible y en consecuencia $A_{11}$ $\neq$ $0$.
Teniendo en cuenta que $A_{11}$ $\neq$ $0$, veamos ahora que al realizar eliminación gaussiana sobre $A$, la matriz $A(1|1)^{(1)}$ resulta ser diagonal dominante:
Hay que probar que para todo $j$ vale:
\[
\sum_{i \geq 2, i \neq j}^{n}|A_{ij}^{(1)}| \leq  |A_{jj}^{(1)}| 
\]
Tenemos que:
\[
\sum_{i \geq 2, i \neq j}^{n} |A_{ij}^{(1)}| =  \sum_{i \geq 2, i \neq j}^{n} |A_{ij} - \frac{A_{1j}A_{i1}}{A_{11}}| \leq 
\sum_{i \geq 2, i \neq j}^{n} |A_{ij}| + |\frac{A_{1j}A_{i1}}{A_{11}}| =
\sum_{i \geq 1, i \neq j}^{n} |A_{ij}| - |A_{1j}| + |\frac{A_{1j}}{A_{11}}|(\sum_{i \geq 2}^{n} |A_{i1}| - |A_{j1}|)
\]
y como A es diagonal dominante:
\[
\sum_{i \geq 1, i \neq j}^{n} |A_{ij}| \leq |A_{jj}| \text{\qquad y \qquad} \sum_{i \geq 2}^{n} |A_{i1}| \leq |A_{11}
\]
Finalmente:
\[
\sum_{i \geq 1, i \neq j}^{n} |A_{ij}| - |A_{1j}| + |\frac{A_{1j}}{A_{11}}|(\sum_{i \geq 2}^{n} |A_{i1}| - |A_{j1}|) \leq
|A_{jj}| - |A_{1j}| + |\frac{A_{1j}}{A_{11}}|(|A_{11} - |A_{j1}|) =
\]

\[
|A_{jj}| - |\frac{A_{1j}A_{j1}}{A_{11}}| \leq
|A_{jj} - \frac{A_{1j}A_{j1}}{A_{11}}| = |A_{jj}^{(1)}|
\]

Por lo tanto $A(1|1)^{(1)}$ es diagonal dominante, y volviendo a utilizar la demostración anterior en cada paso de la eliminación gaussina se concluye que A es posible aplicar eliminación gaussiana sin pivoteo.

