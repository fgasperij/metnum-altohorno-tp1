\subsubsection{Test de isoterma en función de las condiciones de borde}
La siguiente experimentación tiene por objetivo principal analizar cómo se comporta el modelo implementado. 
Es decir, ver cómo refleja los cambios esperados en un situación dada del problema modelado.
En un alto horno real $r_i$ y $r_e$ no sufren cambios, lo que varía constantente son las $T_i(\theta_j)$ y las 
$T_e(\theta_j)$. Para simular esto y poder ver cómo varía la isoterma elegimos una buena discretización, la mayor que 
podamos costear en función de nuestra potencia de procesamiento, mantenemos $r_i$ y $r_e$ constantes y vamos variando 
las $T_i(\theta_j)$ y las $T_e(\theta_j)$ de diferentes formas para ver cómo se comporta la isoterma.
Los casos que vamos a considerar son los siguientes:
\begin{enumerate}
 \item aumenta $T_i(\theta_j)$ y $T_e(\theta_j)$ para todo $j$ de la discretización.
 \item aumenta $T_i(\theta_j)$ y se mantienen constante $T_e(\theta_j)$ para todo $j$ de la discretización.
 \item aumenta $T_i(\theta_j)$ y desciende $T_e(\theta_j)$ para todo $j$ de la discretización.
\end{enumerate}
Elegimos estos tres casos porque creemos que cubren la mayoría de los comportamientos. A nivel teórico no hacemos 
distinción entre las $T_i(\theta_j)$ y las $T_e(\theta_j)$, por lo tanto tenemos dos componentes que pueden cambiar 
en un sentido u otro o mantenerse constantes. El caso 1 tomas que las dos componentes cambian y en el mismo sentido. 
El caso 2 que una componente cambia y la otra permanece constante. Finalmente, el caso 3 considera la situación en la 
que las dos componentes cambian pero en sentidos inversos.
Nuestra hipótesis en esta experimentación es que la isoterma se acercará a la pared que a su vez se acerque al valor
de la isoterma buscada.