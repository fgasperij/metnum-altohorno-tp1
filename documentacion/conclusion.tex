\section{Conclusiones}
%Esta sección debe contener las conclusiones generales del trabajo. Se deben mencionar
%las relaciones de la discusión sobre las que se tiene certeza, junto con comentarios
%y observaciones generales aplicables a todo el proceso. Mencionar también posibles
%extensiones a los métodos, experimentos que hayan quedado pendientes, etc.
Como ya se mencionó, este trabajo tiene dos secciones principales: una es la resolución se sistema de 
ecuaciones lineales mediante técnicas algorítmicas matriciales, y la otra es el problema de estimar una 
isoterma a partir de una cantidad finita de puntos con sus correspondientes temperaturas. Para la parte 
inicial se estudió la eliminación gaussiana y la factorización LU. Pese a que son muy 
parecidas en su definición y en su implementación algorítmica, es preferible usar la segunda en caso
de que se necesite resolver un sistema para una misma matriz $A$ pero con distintos vectores $b$ 
ya que reduce el costo computacional de manera contundente. 

Una desventaja es que la factorización LU no siempre existe pero siempre que exista es
preferible utilizarla antes que la eliminación gaussiana clásica. Más aun, cuanto más vectores 
b distintos haya, mayor será diferencia entre ambos algoritmos. Además, cabe señalar que la factorización LU no requiere espacio de memoria adicional ni agrega complejidad al algoritmo de eliminación gaussiana sin pivoteo.

En el caso de los tests con la isoterma, nuestros algoritmos se acercaron a lo que se esperaría de
la realidad, o la realidad se acercó a lo que se esperaba de nuestros algoritmos, cómo se quiera
verlo. Mientras más ricos seamos en la industria del tiempo de cómputo, más granularidad podremos
permitirnos comprar, y más exacta será nuestra isoterma.

También fue el caso que la isoterma tendía a acercarse a la pared que más debía acercarse según
nuestras predicciones, confirmando empíricamente que nuestras acepciones sobre las leyes físicas y
nuestro entendimiento de la fórmula de expansión del calor eran correctas.

Como trabajo práctico en general, fue una experiencia nueva. Aprendimos a prestarle más atención al
por qué hacíamos las cosas en vez de cómo hacerlas.




