\subsection{Análisis de comparación isoterma en función de las condiciones de borde}

El gráfico \ref{isotermaVsBordesExternaAumenta} muestra claramente que al aumentar la temperatura externa e interna en la misma proporción la isoterma se mueve cada vez más rápido hacia el borde que se acerca a ella. En este caso es el borde externo. La temperatura externa empieza en 0 °C y finaliza en 1000 °C. Esto coincide con lo que esperaríamos en una situación real.
En el gráfico \ref{isotermaVsBordesExternaConstante} mantuvimos la temperatura externa constante por lo tanto la isoterma siguió acercándose al borde exterior pero siempre al mismo ritmo. Interpretamos que se acerca con velocidad lineal porque es la velocidad a la que crece la temperatura interna.
Finalmente, en el gráfico \ref{isotermaVsBordesExternaDesciende} vemos claramente cómo el movimiento de los bordes va afectando directamente a la posición de la isoterma y en sus mismas proporciones. A pesar de que la temperatura externa desciende mientras que la interna asciende, la isoterma sigue acercándose hacia el borde externo. Ésto se debe a que la temperatura externa desciende a un $40\%$ la velocidad a la que aumenta la interna.
Después de analizar éstos tres gráficos concluimos que el modelo implementado coincide en gran parte con lo que el sentido común indica. El valor de la isoterma va modificándose de forma directamente proporcional a el cambio de las temperaturas internas y externas que ejercen influencia sobre ella en sentidos contrarios.